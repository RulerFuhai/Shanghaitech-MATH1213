\documentclass[12pt]{article}
\usepackage{amsmath, amssymb, amsfonts}
\usepackage{geometry}
\geometry{a4paper, margin=1in}
\usepackage{titlesec}
\usepackage{ctex}

\titleformat{\section}{\large\bfseries}{\thesection}{1em}{}
\titleformat{\subsection}{\normalsize\bfseries}{\thesubsection}{1em}{}

\begin{document}
	
	\section*{二:解析函数理论在物理学中的应用}
	
	
	\section{二维电磁场和电动力学}
	
	\subsection{电势和电场的表示}
	在二维静电学中,电势函数 $\phi(x, y)$ 满足拉普拉斯方程:
	\[
	\nabla^2 \phi = 0
	\]
	这意味着 $\phi$ 是一个调和函数。由于调和函数的实部和虚部可以构成一个解析函数,因此可以引入复变量 $z = x + iy$ 和一个解析函数 $f(z)$,使得:
	\[
	f(z) = \phi(x, y) + i\psi(x, y)
	\]
	其中 $\psi(x, y)$ 是电流的流函数。
	
	\subsection{柯西-黎曼条件}
	解析函数的实部和虚部必须满足柯西-黎曼(Cauchy-Riemann)方程:
	\[
	\frac{\partial \phi}{\partial x} = \frac{\partial \psi}{\partial y}, \quad \frac{\partial \phi}{\partial y} = -\frac{\partial \psi}{\partial x}
	\]
	这确保了电场 $\vec{E}$ 和电流密度 $\vec{J}$ 之间的关系可以通过解析函数统一表示。
	
	\subsection{边界值问题的解决}
	利用解析函数,可以将复杂的边界值问题转化为已知的解析函数形式。例如,通过共形映射,可以将复杂的几何形状映射到简单的几何形状,从而简化问题的求解。
	
	\section{流体动力学中的应用}
	
	\subsection{复势函数}
	在二维无粘、不可压缩的流体中,速度场可以用复势函数 $f(z)$ 表示:
	\[
	f(z) = \phi(x, y) + i\psi(x, y)
	\]
	其中 $\phi$ 是速度势,$\psi$ 是流函数。
	
	\subsection{速度场的表示}
	流体的速度 $\vec{v}$ 可以表示为复势函数的导数:
	\[
	\vec{v} = \nabla \phi = \left(\frac{\partial \phi}{\partial x}, \frac{\partial \phi}{\partial y}\right) = \left(\frac{\partial \psi}{\partial y}, -\frac{\partial \psi}{\partial x}\right)
	\]
	
	\subsection{典型应用}
	\begin{itemize}
		\item 分析物体周围的流体流动,例如圆柱或机翼的绕流。
		\item 利用解析函数表示涡流的速度场和势函数。
	\end{itemize}
	
	\section{弹性力学中的应用}
	
	\subsection{复解析函数方法}
	在二维弹性力学中,位移和应力场可以用解析函数表示,特别是利用 Kolosov-Muskhelishvili 方法。
	
	\subsection{应力函数的表示}
	引入两个解析函数 $\phi(z)$ 和 $\psi(z)$,应力分量可以表示为:
	\[
	\sigma_{xx} + \sigma_{yy} = 4\mathrm{Re}[\phi'(z)]
	\]
	\[
	\sigma_{yy} - \sigma_{xx} + 2i\sigma_{xy} = 2[\overline{\psi'(z)} + \phi''(z)]
	\]
	
	\subsection{应用}
	\begin{itemize}
		\item 分析材料中的裂纹尖端应力场。
		\item 解决具有复杂边界条件的弹性问题。
	\end{itemize}
	
	\section{量子力学中的应用}
	
	\subsection{解析延拓和散射理论}
	在量子散射理论中,散射振幅作为复函数,其解析性质对于理解散射过程至关重要。
	\begin{itemize}
		\item 散射振幅的极点:与共振态和束缚态相关。
		\item 解析延拓:用于研究能量的非物理区域。
	\end{itemize}
	
	
	\section{热力学和统计力学}
	
	\subsection{Lee-Yang 零点理论}
	Lee 和 Yang 提出了配分函数零点的理论。在统计力学和统计场论中,杨李定理(或杨李单位圆定理)指出:有些铁磁配分函数的零点都是虚数。通过研究配分函数作为复变量的零点分布,可以理解相变的机制。
	
	\subsection{相变和临界现象}
	\begin{itemize}
		\item 解析奇点:热力学函数在复平面上的奇点决定了物理系统的相变行为。
		\item 临界指数:通过解析函数的行为确定系统的临界指数。
	\end{itemize}
	
	
	
	\section{共形映射}
	
	\subsection{共形映射的性质}
	共形映射保持角度不变,这对于解决具有复杂几何形状的物理问题非常有用。
	
	\subsection{应用}
	\begin{itemize}
		\item 电磁学:将复杂边界条件的问题映射为简单几何形状的问题。
		\item 流体力学:解决具有复杂边界的流动问题。
	\end{itemize}
	
	\subsection{典型映射}
	\begin{itemize}
		\item Joukowski 变换:茹科夫斯基变换(英语:Joukowsky transform)是一种用于翼型设计的共形映射,以俄罗斯科学家尼古拉·叶戈罗维奇·茹科夫斯基的名字命名在空气动力学中,茹科夫斯基变换可以用来求解绕茹科夫斯基翼型的二维势流。茹科夫斯基翼型的后缘处为一尖点。茹科夫斯基变换可以看成是卡门-特雷夫茨变换(Kármán-Trefftz transform)的特例。卡门-特雷夫茨翼型的后缘角是可变的,当后缘角为0时,即是茹科夫斯基翼型。。
		\item Schwarz-Christoffel 变换:施瓦茨—克里斯托费尔(Schwarz-Christoffel)映射是复平面的变换,把上半平面共形地映射到一个多边形。施瓦茨—克里斯托费尔映射可用在位势论和其它应用,包括极小曲面和流体力学中。施—克映射有一个缺陷,它无法较好的处理不规则几何图形和有孔的情况,这个问题已被伦敦皇家学院应用数学教授Darren Crowdy解决。
	\end{itemize}
	
	
\end{document}
